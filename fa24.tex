\documentclass{amsart}

\usepackage{notes}

\title{Student Arithmetic Geometry Seminar: On the Birational Geometry of Stacks}
\date{\today}

\begin{document}

\begin{abstract}
	These are my notes from the Fall 2024 student arithmetic geometry seminar.
	This is a papers seminar focused on birational geometry and stacks.
	I make no promises about the quality of the notes, but feel free to bring to my attention anything that could be improved.
\end{abstract}

\maketitle

\section{8/30 (Martin Olsson) -- ???}

I missed this day.
If you have notes you would like to share, please send them to me and I will \TeX\, them up.
(Alternatively, feel free to do so yourself and submit a pull request!)

\section{9/6 (Martin Olsson) -- Algebraic Stacks through Examples}

Our goal this time is to give a more precise discussion of stacks.
Throughout we fix a base scheme $S$.

\subsection{Working definition of algebraic stacks}

\begin{dfn}
	An \emph{algebraic stack} is a functor $\Xfr: (\Sch / S)\op \to \Grpd$ such that:
	\begin{enumerate}
		\item $\Xfr$ is a sheaf for the \'etale topology (i.e.\ satisfies descent),
		\item the diagonal $\Delta_\Xfr$ is representable, and 
		\item $\Xfr$ admits a smooth cover by a scheme.
	\end{enumerate}
\end{dfn}

Making this precise takes some work -- it is often technically easier to work with groupoids rather than fibered categories.
The first condition gives the notion of a \emph{stack} -- algebraicity corresponds to the second and third conditions.

\begin{ex}
	Let $U$ be a scheme over $S$ and let $G$ be a flat affine $S$-group scheme.
	Then we can construct a quotient stack $[U/G]$, defined below.
\end{ex}

\subsection{Principal homogeneous spaces and torsors}

Let $G$ be an affine group scheme over $S$.

\begin{dfn}
	A \emph{principal homogeneous space} under $G$ is a flat surjective $S$-scheme $P$ with left $G$-action such that the map 
	\begin{align*}
		G \times_S P &\to P \times_S P \\
		(g, x) &\mapsto (gx, x)
	\end{align*}
	is an isomorphism.
\end{dfn}

\begin{ex}
	There is an equivalence between the groupoid of invertible sheaves on $S$ and the groupoid of $\GG_m$-principal homogeneous spaces on $S$.
	Given a line bundle $\Lc$, the corresponding homogeneous space is $\Isom(\Lc, \Oc) = \Spec_S \oplus_{n \in \ZZ} \Lc^{\otimes n}$.
	Here $u \in \GG_m$ acts on $\oplus_n \Lc^{\otimes n}$ via $u \cdot \ell^{\otimes n} = u^n \ell^{\otimes n}$.
\end{ex}

Let $\Xfr = [U / G]$.
Then $\Xfr(T)$ is defined to be the groupoid of commutative diagrams
\[
	\begin{tikzcd}
		P \rar["\rho"] \dar & U \dar \\
		T \rar & S
	\end{tikzcd}
\]
where $P \to T$ is a principal $G$-homogeneous space and $\rho$ is $G$-equivariant.
This is an algebraic stack!

Descent follows from faithfully flat descent for affine schemes.
For the second and third conditions, let $(P, \rho), (P', \rho') \in \Xfr(T)$.
Define $\Isom_{\Xfr}((P, \rho), (P', \rho')): (\Sch_{/T})\op \to \Set$ by 
\[
	\Isom_{\Xfr}((P, \rho), (P', \rho'))(V) = \left\{ \sigma: P_V \to P'_V \big| \sigma \textrm{ is an isomorphism over $V$ and } \rho' \circ \sigma = \rho \right\}.
\]
The second condition is saying that $\Isom_{\Xfr}((P, \rho), (P', \rho'))$ is representable by a scheme.
If $G$ is smooth, then the third statement is saying that for the universal homogeneous space $(P_0, \rho_0) \in \Xfr(U)$, if we are given $(P, \rho)$ on $T$, then the map $\Isom_{\Xfr}((P_0, \rho_0), (P, \rho)) \to T$ is a smooth surjection.

\begin{ex}
	Consider the case $U = S$ and $G = \GG_m$.
	Here a principal $\GG_m$-homogeneous space $P$ corresponds to a line bundle $\Lc$.
	We have $\Isom_T(P, P') \simeq \Isom_T(\Lc', \Lc) \simeq \Isom_T(\Lc' \otimes \Lc^\vee, \Oc)$.
	This is representable.
\end{ex}

\begin{ex}
	Let $k$ be a field of characteristic $p$, and let $G = \mu_p$.
	Note that $\mu_p$ is flat but not smooth.
	Write $B\mu_p = [\Spec k / \mu_p]$.
	From the short exact sequence
	\[
		\SES{\mu_p}{\GG_m}{\GG_m}
	\]
	we see that the groupoid of $\mu_p$-principal homogeneous spaces on a test scheme $T$ is equivalent to the category of pairs $(\Lc, \lambda)$ where $\lambda: \Lc^{\otimes p} \xrightarrow{\sim} \Oc$.
	Specifically, the principal homogeneous space corresponding to $(\Lc, \lambda)$ is $\Spec_T \oplus_{i=0}^{p-1} \Lc^{\otimes i}$.
	It follows that $B \mu_p \simeq [\GG_m /_p \GG_m]$, where the $/_p$ indicates $\GG_m$ acting on itself via the $p$th power map.
	We have taken a ``flat algebraic stack'' and replaced it by a genuine (smooth) algebraic stack!
	A theorem of Artin allows us to do this more generally.
\end{ex}

\begin{ex}
	Let $\Xfr$ be a ``stacky $\PP^1$'' with stabilizer $\mu_2$ at $z = 0$ and $\mu_3$ at $z = \infty$.
	We may construct this as a stack over $\PP^1$ with $T$-points (for $g: T \to \PP^1$)
	\[
		\Xfr(T) = \{ (\Lc_0, \alpha_0: \Lc_0 \to \Oc_T, \lambda_0: \Lc_0^{\otimes 2} \xrightarrow{\sim} g^* \Ic_0, \textrm{ likewise at $\infty$}) \,|\, \textrm{relevant diagrams commute}\}.
	\]
	Here the diagram for $0$ asserts that $\alpha_0^2$ equals the composite of $\lambda_0$ and the inclusion $g^* \Ic_0 \hookrightarrow \Oc_T$.
	The diagram for $\infty$ is similar but involves $\alpha_\infty^3$.

	This is not obviously a quotient stack $[U / G]$.
	In fact, it is impossible to write $\Xfr = [U / G]$ for a finite discrete group $G$: away from $0$ and $\infty$, the map $U \to \Xfr$ would be a finite \'etale cover, which would have to be an $n$-fold multiplication map $\GG_m \to \GG_m$ for some $n$.
	But there's no possible choice of $n$ that works (we have a two-fold cover at $0$ and a three-fold cover at $\infty$).

	However, we can realize $\Xfr = [(\AA^2 \setminus (0, 0)) / \GG_m]$ as a weighted projective space (where $\GG_m$ acts on the first coordinate by weight two and on the third coordinate by weight three).
\end{ex}

\section{9/13 (Martin Olsson) -- More Examples}

\subsection{Stacky projective lines}

We begin with the example from last time.

\begin{ex}
	Consider again the stacky projective line $\Xfr$ with notation as above.
	One thinks of $\Lc_0$ as a ``square root'' of the ideal sheaf $\Ic_0$ and $\Lc_\infty$ as a ``square root'' of $\Ic_\infty$.
	Away from the points $\{0, \infty\}$, $\Xfr$ looks like $\PP^1$.
	The fiber of $\Xfr$ over $0 \in \PP^1$ is $(\Spec k[\epsilon] / (\epsilon^2)) / \mu_2$, where $\mu_2$ acts by $\epsilon \mapsto -\epsilon$.

	We can give multiple presentations of $\Xfr$.
	The typical approach to do so is to ``rigidify'' $\Xfr$ by choosing additional data and then quotienting by the choice of additional data.
	\begin{enumerate}
		\item Let's rigidify $\Xfr$ by fixing isomorphisms $\Lc_0 \simeq \Oc$ and $\Lc_\infty \simeq \Oc$.
			Then $\alpha_0, \alpha_\infty \in \Gamma(T, \Oc_T)$, so we may view these as functions $f_0, f_\infty$.
			The $\lambda$'s are replaced by $\gamma_0$, $\gamma_\infty$, where $\gamma_i$ trivializes $g^* \Ic_i$.
			We may then define a scheme $U$ over $\PP^1$ by
			\[
				U(T) = \{ (g: T \to \PP^1, f_0, f_\infty, \gamma_0, \gamma_\infty) \,|\, \textrm{ the necessary diagrams commute} \}.
			\]
			This $U$ is a scheme: we may view it explicitly as a closed subscheme of $(\PP^1 \times \AA^2) \times_{\PP^1} \Isom(\Oc, \Ic_0) \times_{\PP^1} \Isom(\Oc, \Ic_\infty)$ (cut out by the condition that the diagrams commute).
			We have an action of $\GG_m^2$ on $U$:
			\[
				(u_0, u_\infty) \cdot (g, f_0, f_\infty, \gamma_0, \gamma_\infty) = (g, u_0 f_0, u_\infty f_\infty, u^2_0 \gamma_0, u_\infty^3 \gamma_\infty)
			\]
			Thus we may construct the quotient $[U / \GG_m^2]$.

			We claim that $\Xfr \simeq [U / \GG_m^2]$.
			In fact, the point $(g, (\Lc_0, \alpha_0), (\Lc_0, \alpha_\infty), \lambda_0, \lambda_\infty)$ corresponds to the $\GG_m^2$-torsor $\Isom(\Oc, \Lc_0) \times_T \Isom(\Oc, \Lc_\infty)$ together with its natural map to $U$.
		\item There are other presentations!
			This is an advantage of stacks (as opposed to just considering group actions).
			We can write $\Xfr \simeq [(\AA^2 \setminus \{ 0 \}) / \GG_m]$ where $u \in \GG_m$ acts by $u \cdot (s, t) = (u^2 s, u^3 t)$.
			Note that $T$-points of $[(\AA^2 \setminus \{ 0 \}) / \GG_m]$ are 
			\[
				\{ (\Lc, a \in H^0(T, \Lc^{\otimes 2}), b \in H^0(T, \Lc^{\otimes 3})) \,|\, \textrm{$a$, $b$ not simultaneously $0$} \}.
			\]

			To see the equivalence between this and $\Xfr$, fix an isomorphism $\Ic_0 \simeq \Ic_\infty$.
			This forces $\Lc_0^2 \simeq \Lc_\infty^3$, and we can compute $(\Lc_0 \otimes \Lc_\infty\inv)^{\otimes 2} \simeq \Lc_\infty$ and $(\Lc_0 \otimes \Lc_\infty\inv)^{\otimes 3} \simeq \Lc_0$.
			Letting $\Mc = \Lc_0\inv \otimes \Lc_\infty$, we see that $\alpha_0 \in \Mc^{\otimes 2}$ and $\alpha_\infty \in \Mc^{\otimes 3}$.
			Using this gives the equivalence.
	\end{enumerate}
\end{ex}

\subsection{Weighted projective spaces}

We can generalize the latter construction.

\begin{ex}
	Let $\GG_m$ act on $\AA^{n+1}$ by $u \cdot (x_0, \dots, x_n) = (u^{\alpha_0} x_0, \dots, u^{\alpha_n} x_n)$.
	Removing the origin gives the weighted projective stack $\PP(\alpha_0, \dots, \alpha_n) = [(\AA^{n+1} \setminus \{0\}) / \GG_m]$.
	There are many interesting sub-examples:
	\begin{enumerate}
		\item $\PP^n = \PP(1, \dots, 1)$
		\item We have $\ol{\Mfr}_{1,1} \simeq \PP(4, 6)$ (though we will not show this here).
		\item $\PP(1, -1)$ has $T$-points 
			\[
				\{ (\Lc, \alpha_+ \in H^0(T, \Lc), \alpha_- \in H^0(T, \Lc\inv)) \,|\, \textrm{$a_+$, $a_-$ not simultaneously zero } \}.
			\]
			This is covered by $U_+$ and $U_-$ where 
			\[
				U_\pm(T) = \{ (\Lc, a_+, a_-) \,|\, \textrm{$a_\pm$ generates $\Lc^{\otimes \pm 1}$} \}.
			\]
			The intersection is $\GG_m$, and $\PP(1, -1)$ is the affine line with two origins.
	\end{enumerate}
\end{ex}

This has some interesting applications.

\begin{ex}
	Consider the polynomial $x^2 + 29y^2 - 3z^3 \in \ZZ[x, y, z]$.
	We might like to consider a ``projective variety'' corresponding to this, but attempting to do so runs into an obvious issue: the polynomial is not homogeneous!
	We can obtain instead a ``weighted projective variety'' 
	\[
		[(\Spec \ZZ[x, y, z] / (x^2 + 29y^2 - 3z^3) \setminus \{0\}) / \GG_m] \subset \PP(3, 3, 2).
	\]
	This is proper over $\Spec \ZZ$.
	See a paper of Bhargava and Poonen for more discussion of this in the context of finding rational points on curves.
	A classic result of Darmon and Granville shows that this fails local-to-global (rational points exist locally but not globally).
\end{ex}

\section{9/20 (Rose Lopez) -- Birational Geometry of Stacks}

References for today's talk are:
\begin{itemize}
	\item Kresch and Tschinkel (2023) -- \textit{Birational Geometry of Deligne-Mumford Stacks}
	\item Kresch and Tschinkel (2019) -- \textit{Birational Types of Algebraic Orbifolds}
	\item Kontsevich and Tschinkel (2017) -- \textit{Specialization of Birational Types}
	\item Bergh and Rydh (2019) -- \textit{Functorial Destackification}
\end{itemize}

\subsection{Classical Birational Geometry}

\begin{dfn}
	Let $X$ and $Y$ be varieties.
	A \emph{rational map} $f: X \dashrightarrow Y$ is a morphism $f: U \to Y$ for some dense open $U \subset X$.
	We identify two such maps $f: U \to Y$, $f': U' \to Y$ if $f|_{U \cap U'} = f'|_{U \cap U'}$.

	We say that $f: X \dashrightarrow Y$ is \emph{birational} if there exists an inverse map $Y \dashrightarrow X$.
	This induces an isomorphism on dense opens ($U \xrightarrow{\sim} V$) as well as an isomorphism on function fields.
\end{dfn}

\begin{thm}[Weak Factorization]
	Let $f: X \to Y$ be a birational map between smooth complete varieties over an algebraically closed field $k$ of characteristic zero.
	Let $U \subset X$ be an open such that $f|_U$ is an isomorphism onto its image.
	Then $f$ factors as a zigzag of blowups at smooth centers inducing isomorphisms on the image of $U$ (??).
\end{thm}

\subsection{The Burnside Ring}

Fix an algebraically closed field $k$ of characteristic zero.

\begin{dfn}
	Let $\Burn_n$ be the free abelian group generated by isomorphism classes of finitely generated fields of transcendence degree $n$. 
	For a smooth projective irreducible variety $X$ of dimension $n$, we define $[X] := [k(X)] \subset \Burn_n$.
	The \emph{Burnside ring of varieties} is $\Burn = \oplus_n \Burn_n$ with ring structure given by $[X] \cdot [Y] = [X \times Y]$.
\end{dfn}

For $U = X \setminus D$ where $D = D_1 \cup \dots D_\ell$ is a simple normal crossing divisor, we define
\[
	[U] := [X] - \sum_i [D_i \times \PP^1] + \sum_{i < j} [(D_i \cap D_j) \times \PP^2]
\]
We also let $[X \cup Y] = [X] + [Y]$.
Note that $[U] = [U']$ if there exists a quasiprojective variety $V$ and a pair of birational projective morphisms $V \to U$ and $V \to U'$.

\begin{prop}
	The group $\Burn_n$ is generated by classes $[U]$ where $U$ is $n$-dimensional and quasiprojective, modulo the \emph{modified scissor relation}
	\[
		[U] = [V \times \PP^{r-d}] + [U \setminus V]
	\]
	for any smooth closed $V \subset U$ of dimension $< n$.
\end{prop}

\begin{prop}[Specialization]
	Let $\pi: X \to B$ and $\pi': X' \to B$ be smooth proper morphisms to a smooth connected curve $B$.
	If the generic fibers of $\pi$ and $\pi'$ are birational over $k(B)$, then for any closed $b \in B$, the fibers $\pi\inv(b)$ and $(\pi')\inv(b)$ are birational over $k(b)$.
\end{prop}

\subsection{Extension to stacks}

\begin{dfn}
	Let $\Xfr$ and $\Yfr$ be stacks.
	A rational map $f: \Xfr \dashrightarrow \Yfr$ is a morphism defined on a dense open $\Ufr \subset \Xfr$.
	A 2-isomorphism $\alpha: (U, f) \to (U', f')$ is an isomorphism $f|_{U \cap U'} \xrightarrow{\sim} f'|_{U \cap U'}$.
	We say that $f$ is birational if it induces an equivalence on dense opens.
\end{dfn}

\begin{dfn}
	Let $\Xfr$ and $\Yfr$ be stacks.
	We say that $\Xfr$ and $\Yfr$ are \emph{birationally equivalent} if there exists a representable proper birational map $\Xfr \dashrightarrow \Yfr$, or equivalently a span $\Xfr \leftarrow \Xfr' \rightarrow \Yfr$ consisting of representable proper birational morphisms.
\end{dfn}

\begin{ex}
	Let $\Xfr$ be a stack and $\Zfr$ a closed substack.
	We can define a \emph{blowup} $\Bl_\Zfr \Xfr \to \Xfr$ by performing the usual scheme-theoretic blowup locally and gluing.
\end{ex}

\begin{ex}
	Let $\Xfr$ be a smooth separated DM stack and $\Dfr \subset \Xfr$ a divisor.
	We can define a \emph{root stack} $\sqrt[r]{(\Xfr, \Dfr)} \to \Xfr$ by
	\[
		\sqrt[r]{(\Xfr, \Dfr)}(T) = \big\{ (g: T \to \Xfr, \alpha: \Lc \to \Oc_T, \lambda: \Lc^{\otimes r} \xrightarrow{\sim} g^* \Ic_{\Dfr} ) \,\big|\, \textrm{the induced maps $\Lc^{\otimes r} \to \Oc_T$ agree}\big\}
	\]
	The morphism $\sqrt[r]{(\Xfr, \Dfr)} \to \Xfr$ is proper but not representable: the fibers over points in $\Dfr$ are stacky.
\end{ex}

\begin{ex}
	Let $\Xfr$ be a stack and $\Lc$ a line bundle on $\Xfr$.
	The \emph{gerbe of $n$th roots of $\Lc$} is $\sqrt[n]{\Lc / \Xfr} \to \Xfr$ with
	\[
		\sqrt[n]{\Lc / \Xfr}(T) = \big\{ (g: T \to \Xfr, \Mc \in \Pic T, \epsilon: \Mc^{\otimes n} \xrightarrow{\sim} g^* \Lc) \big\}
	\]
	We can contrast this with the above example: essentially, the above example rigidifies things by considering $\Ic_\Dfr \to \Oc$.
	Over $\Ufr = \Xfr \setminus \Dfr$, the gerbe $\sqrt[n]{\Lc / \Xfr}|_{\Ufr}$ is trivializable.
\end{ex}

There are two weak factorization results for stacks.
One factorizes a representable proper birational map into blowups and can be used to define the ``correct'' (?) Burnside ring.
Another factorizes a proper birational map into stacky blowups and can be used to define a weaker Burnside ring related to ``$G$-equivariant Burnside rings.''

\begin{ex}
	Let $C = \AA^1$, $C_0 = \AA^1 \setminus \{0\}$, $E$ an elliptic curve, and $p \in E$.
	Define $\Xfr_0 = C_0 \times E \times B(\ZZ / 2) \subset \Xfr_1 = C \times E \times B(\ZZ / 2)$.
	Let $B = \Bl_{(0,p)} (C \times E)$ and $\Xfr_2 = \sqrt{\Oc_B(D)/B}$.
	We get an inclusion $\Xfr_0 \to \Xfr_2$.
	This induces an equivalence on generic fibers but not on special fibers (over $(0, p)$ ??).
	The ``specialization'' result from earlier fails for stacks!
\end{ex}

\section{9/27 (Xiangru Zeng) -- Valuative Criteria for the Existence of Moduli Spaces}

Let $k$ be an algebraically closed field of characteristic zero.
We may skip some less-than-essential details.

References are:
\begin{itemize}
	\item Alper -- Good Moduli Spaces for Artin Stacks
	\item Alper, Halpern-Leistner, Heinloth -- Existence of Moduli Spaces for Algebraic Stacks
	\item Alper -- Notes on Stacks and Moduli
\end{itemize}

\subsection{Precursors}

\begin{thm}[Keel-Mori]
	Let $\Xfr$ be a separated Deligne-Mumford stack of finite type over a base $S$.
	Then there exists coarse moduli space $\pi: \Xfr \to X$, i.e.\ a universal algebraic space $X$ among algebraic spaces equipped with a map from $X$ and such that $\pi$ is bijective on geometric points.
	This satisfies $\pi_* \Oc_\Xfr = \Oc_X$.
	In characteristic zero, we end up with a ``tame moduli space:''
	\begin{enumerate}
		\item If $\Xfr \to S$ is flat, then $X \to \Spec k$ is flat.
		\item The forgetful functor $\pi_*: \QCoh(\Xfr) \to \QCoh(X)$ is exact.
	\end{enumerate}
\end{thm}

This requires $\Xfr$ to be separated -- in particular, the stabilizer groups of $\Xfr$ must be affine.
This is a relatively strict condition!

When considering problems of GIT, we often let $U \subset \PP^n$ be a projective variety with linearized $G$-action (for some linearly reductive group $G$).
The GIT quotient is then
\[
	U \sslash G = \Proj \oplus_{n \geq 0} \Gamma(U, \Oc_U(n))^G.
\]
We have a map $\pi: [U^{ss} / G] \to U \sslash G$, where $U^{ss}$ is the semistable locus.
This satisfies:
\begin{enumerate}
	\item $\Oc(U \sslash G) = \pi_*(\Oc(U^{ss}))^G$, and
	\item (In characteristic zero) $\pi$ is affine.
\end{enumerate}
Seshadri called a map satisfying the above conditions a \emph{good quotient}.

\subsection{Good moduli spaces}

\begin{dfn}
	A map $\pi: \Xfr \to X$, where $\Xfr$ is a (qcqs?) algebraic stack and $X$ is a (qcqs?) algebraic space, is a \emph{good moduli space} if
	\begin{enumerate}
		\item $\pi_* \Oc_\Xfr \cong \Oc_X$, and
		\item $\pi_*: \QCoh(\Xfr) \to \QCoh(X)$ is exact.
	\end{enumerate}
\end{dfn}

\begin{thm}
	Let $\pi: \Xfr \to X$ be a good moduli space.
	Then:
	\begin{enumerate}
		\item $\pi$ is surjective and universally closed.
		\item For $x_1, x_2 \in \Xfr(k)$, then $\pi(x_1) = \pi(x_2)$ if and only if $\ol{\{x_1\}} \cap \ol{\{x_2\}} \neq \varnothing$.
		\item If $\Xfr$ is noetherian, then $\pi$ is universal among maps from $\Xfr$ to algebraic spaces.
	\end{enumerate}
\end{thm}

\begin{ex}
	Let $G$ be a linearly reductive group acting on an affine scheme $\Spec A$.
	Then $[A / G] \to \Spec A^G$ is a good moduli space.
	In particular, $[\AA^n / \GG_m] \to \Spec k$ is a good moduli space.
	However, if we remove the origin, then $[(\AA^n \setminus \{0\}) / \GG_m] \to \PP^{n-1}$ is a good moduli space (in fact, the map is an equivalence).
\end{ex}

In positive characteristic, the notion of a good moduli space is too restrictive.
People consider ``adequate moduli spaces'' instead.
We won't discuss these in detail.

\subsection{Existence of good moduli spaces}

\begin{thm}[Alper, Halpern-Leistner, Heinloth]
	Let $\Xfr$ be an algebraic stack of finite type with affine diagonal.
	Then $\Xfr$ admits a separated good moduli space if and only if
	\begin{enumerate}
		\item $\Xfr$ is $\Theta$-reductive, and 
		\item $\Xfr$ is $S$-complete.
	\end{enumerate}
\end{thm}

Let's explain what these mean.
Let $\Theta = [\AA^1 / \GG_m]$, and for any ring $A$, we write $\Theta_A = \Theta \times_{\Spec k} \Spec A$.
For a DVR $R$ with fraction field $K$, let $\Theta_R \setminus \{0\} = \Spec R \cup_{\Spec K} \Theta_K$.

\begin{dfn}
	Say $\Xfr$ is $\Theta$-reductive if, for all DVRs $R$ and all $\Theta_R \setminus \{0\}$, there exists a unique extension
	\[
		\begin{tikzcd}
			\Theta_R \setminus \{0\} \rar \dar & \Xfr. \\ 
			\Theta_R \ar[ur, dashed, "\exists !", swap] &
		\end{tikzcd}
	\]
\end{dfn}

For $S$-completeness, if $R$ is a DVR with uniformizer $\pi$, let
\[
	\phi_R = [\Spec R[s, t] / (st - \pi) / \GG_m]
\]
where $\GG_m$ acts by weight 1 on $s$ and weight $-1$ on $t$.

\begin{dfn}
	Say $\Xfr$ is $S$-complete if, for all DVRs $R$, there exists a unique extension
	\[
		\begin{tikzcd}
			\phi_R \setminus \{0\} \rar \dar & \Xfr. \\
			\phi_R \ar[ur, dashed, "\exists !", swap] &
		\end{tikzcd}
	\]
\end{dfn}

\begin{ex}
	If $U = \Spec A$ is acted on by $G$, then $[U / G]$ is $\Theta$-reductive if, whenever $\lambda: \GG_m \to G$ and $\pi: \Spec R \to U$ are maps (with $R$ a DVR) such that $\lim_{t \to 0} \lambda(t) \pi(\Spec K)$ exists, then $\lim_{t \to 0} \lambda(t) \pi(\Spec R)$ also exists.
	We can interpret $S$-completeness similarly.
\end{ex}

The theorem is proved by first using the existence of good moduli spaces for quotient stacks $\Spec A / G$
In extending this result to more general $\Xfr$, we use the local structure theorem to write $\Xfr$ in an \'etale neighborhood of a point $x$ as $[\Spec A / G_x]$, where $G_x$ is the automorphism group of $x$.
For gluing, we consider a diagram
\[
	\begin{tikzcd}
		{[\Spec B / G_x]} \ar[r, shift right] \ar[r, shift left] \dar & {[\Spec A / G_x]} \rar \dar & \Xfr \\
		\Spec B^{G_x} \ar[r, shift right] \ar[r, shift left] & \Spec A^{G_x} \rar & X.
	\end{tikzcd}
\]
We can use Luna's fundamental theorem and ``$\Theta$-surjectivity'' to glue things.

\section{10/4 (Will Fisher) -- Smoothness of the Logarithmic Hodge Moduli Space}

The reference for today is:
\begin{itemize}
	\item De Cataldo, Herrero, Zhang -- Geometry of the Logarithmic Hodge Moduli Space
\end{itemize}

We'll begin by reviewing this moduli space and why we should care about it.

\subsection{Review of Hodge theory}

If $X$ is a topological space and $\Fc$ is a sheaf of abelian groups on $X$, then we can define the sheaf cohomology $H^\bullet(X, \Fc)$ by taking an injective resolution $0 \to \Fc \to \Ic^\bullet$ and setting
\[
	H^k(X, \Fc) := H^k(\Gamma(X, \Ic^\bullet)).
\]
More generally, if $\Fc^\bullet$ is a chain complex, we define the \emph{hypercohomology} $\HH^\bullet(X, \Fc)$ by choosing a quasi-isomorphism $\Fc \xrightarrow{\sim} \Ic^\bullet$ with $\Ic^\bullet$ a complex of injectives and setting
\[
	\HH^k(X, \Fc^\bullet) := H^k(\Gamma(X, \Ic^\bullet)).
\]

For Hodge theory, let $X$ be a smooth projective variety over $\CC$.
By GAGA and the $(p, q)$-form decomposition, we have
\[
	\HH^k(X, (\Omega_X^\bullet, d)) \cong H^k_{\dR}(X^\an, \CC).
\]
Formal manipulations give
\begin{align*}
	\HH^k(X, (\Omega_X^\bullet, 0)) &= \HH^k\big(X, \oplus_i \Omega_X^i [-i]\big) \\
					&= \oplus_i \HH^k\big(X, \Omega_X^i [-i]\big) \\
					&= \oplus_i \HH^{k-i}(X, \Omega_X^i) \\
					&= \oplus_{p+q=k} \HH^q(X, \Omega_X^p).
\end{align*}
Thus the usual Hodge decomposition can be rewritten as
\[
	\HH^k(X, (\Omega_X^\bullet, d)) \cong \HH^k(X, (\Omega_X^\bullet, 0)).
\]
This suggests that we should think of the Hodge decomposition as a statement about \emph{sheaves} rather than as a statement just about $X$.
Let's try to understand how to generalize this.

\subsection{Non-abelian Hodge theory}

Let $\Fc$ be a vector bundle on $X$ (still assumed to be a smooth complex projective variety).
Suppose we have a connection $\nabla: \Fc \to \Fc \otimes \Omega^1_X$.
Recall that this means $\nabla$ satisfies the Leibniz rule
\[
	\nabla(fs) = s \otimes df + f \nabla s
\]
for $f \in \Oc_X$, $s \in \Fc$.

If $\nabla$ is flat, i.e. $\nabla \circ \nabla = 0$, then we get an associated complex
\[
	\begin{tikzcd}
		0 \rar & \Fc \rar["\nabla"] & \Fc \otimes \Omega^1_X \rar["\nabla"] & \Fc \otimes \Omega^2_X \rar["\nabla"] & \dots
	\end{tikzcd}
\]
We can (and should) view $(\Omega_X^\bullet, d)$ as the complex associated to $\Oc_X$ with the flat connection $d$.

For the other side of the non-abelian Hodge theorem, we need a new definition.

\begin{dfn}
	Let $\Fc$ be a vector bundle on $X$.
	A \emph{Higgs field} is an $\Oc_X$-linear map $\phi: \Fc \to \Fc \otimes \Omega^1_X$ satisfying $\phi \wedge \phi = 0$.
	We call the pair $(\Fc, \phi)$ a \emph{Higgs bundle}.
\end{dfn}

The Higgs field condition can be understood locally: writing $\phi = \sum_i A_i dx_i$, we require
\[
	\phi \wedge \phi := \sum_{i < j} [A_i, A_j] dx_i \wedge dx_j = 0.
\]

If $(\Fc, \phi)$ is a Higgs bundle, we get a complex
\[
	\begin{tikzcd}
		0 \rar & \Fc \rar["\phi"] & \Fc \otimes \Omega^1_X \rar["\phi"] & \Fc \otimes \Omega^2_X \rar["\phi"] & \dots
	\end{tikzcd}
\]
In particular, $(\Omega_X^\bullet, 0)$ is the complex associated to the Higgs bundle $(\Oc_X, 0)$.

\begin{rmk}
	Arthur Ogus suggested an alternative viewpoint: we can think of the Higgs field structure as a suitable lift of $\Fc$ to the cotangent space $T^* X$.
\end{rmk}

We can now state the non-abelian Hodge theorem.

\begin{thm}[Non-abelian Hodge theorem]
	Let $X$ be a smooth projective variety over $\CC$.
	Then (up to stability conditions) there exists a ``cohomology preserving'' correspondence between:
	\begin{enumerate}
		\item The moduli space $\Mflat$ of flat bundles on $X$.
		\item The moduli space $\MHiggs$ of Higgs bundles on $X$.
	\end{enumerate}
	This correspondence sends $(\Oc_X, d)$ to $(\Oc_X, 0)$.
\end{thm}

The non-abelian Hodge theorem is essentially analytic.
However, we can hope for weaker algebraic statements: perhaps we can find relationships between the corresponding moduli spaces.

One result in this direction is a cohomology equivalence: $H^\bullet(\Mflat) \cong H^\bullet(\MHiggs)$.
This can be shown by constructing a suitable geometric family interpolating between the two moduli spaces.

\subsection{Construction of the interpolation}

Let $X$ be a smooth scheme over a base $B$.

\begin{dfn}
	Given a vector bundle $\Fc$ on $X$ and $t \in \Gamma(B, \Oc_B)$, a \emph{$t$-connection} on $\Fc$ is an $\Oc_B$-linear map $\nabla: \Fc \to \Fc \otimes \Omega_{X/B}^1$ satisfying the $t$-twisted Leibniz rule:
	\[
		\nabla (fs) = ts \otimes df + f \nabla s
	\]
	for $f \in \Oc_X$ and $s \in \Fc$.
	We can use the Leibniz rule to extend $\nabla$ to $\Fc$-valued $n$-forms.
	This allows us to say that $\nabla$ is \emph{flat} if $\nabla \circ \nabla = 0$.
\end{dfn}

\begin{ex}
	If $t = 1$, a $1$-connection is just a connection.
\end{ex}

\begin{ex}
	If $t = 0$, the $0$-twisted Leibniz rule is $\nabla(fs) = f \nabla(s)$, so a flat $0$-connection is just a Higgs field.
\end{ex}

The general goal is to use GIT to construct a moduli space $(\MHodge)_X$ of $t$-connections on $X$ together with a smooth map 
\begin{align*}
	(\MHodge)_X &\to \AA^1_B \\
	(\Fc, \nabla_t) &\mapsto t.
\end{align*}

We'll restrict our hypotheses to make this tractable.
In particular, we assume:
\begin{itemize}
	\item $B$ is a noetherian scheme.
	\item $C$ is a smooth proper scheme over $B$ with geometrically integral fibers of dimension $1$.
	\item $D \hookrightarrow C$ is a relative Cartier divisor such that every geometric fiber over $B$ is reduced and non-empty.
	\item The rank $n$ and degree $d$ are coprime.
\end{itemize}

We'll write $C_B$ for $C$ as a $B$-scheme and $C_S$ for a base change $C_B \times_B S$.

\begin{dfn}
	Let $(\MHodgefr)_{C_B} \to \AA^1_B$ be the moduli stack of rank $n$, degree $d$ flat $t$-connections.
	The groupoid of $S$-points (for $S \to \AA^1_B$) consists of pairs $(\Fc, \nabla)$ where:
	\begin{enumerate}
		\item $\Fc$ is a vector bundle of rank $n$ on $C_S$ such that the restriction to each geometric fiber has degree $d$.
		\item $\nabla: \Fc \to \Fc \otimes \omega_{C_S / S}(D_S)$ is a flat $t$-connection.
	\end{enumerate}
	Taking GIT quotients, we obtain a moduli space $(\MHodge)_{C_B} \to \AA^1_B$.
\end{dfn}

\subsection{Smoothness}

Our goal in the remaining time is to show $(\MHodge)_{C_B} \to \AA^1_B$ is smooth.
We proceed in a few steps:
\begin{enumerate}
	\item The map $(\MHodgefr)_{C_B} \to (\MHodge)_{C_B}$ is a smooth surjection, in fact a smooth good moduli space in the sense of Alper.
		This allows us to reduce to checking that $(\MHodgefr)_{C_B} \to \AA^1_B$ is smooth.
		This is where the coprimality hypothesis arises.
	\item Then we develop an obstruction theory for $t$-connections, i.e.\ an obstruction module $\Qc_x$ for every $x: \Spec A \to (\MHodgefr)_{C_B}$ which is compatible with base change.
		This module controls lifts along square-zero extensions.
	\item This allows us to reduce to showing that $\Qc_x$ vanishes for every geometric point $x: \Spec k \to (\MHodgefr)_{C_B}$.
	\item Given such a geometric point $x$, we extend this to a family $\AA^1_k \to (\MHodgefr)_{C_B}$ which is compatible with the rescaling of $t$-connections and is such that $x$ corresponds to $1 \in \AA^1_k$.
	\item Degenerating our $t$-connection along this family allows us to reduce to showing that $\Qc_x = 0$ for geometric points $x$ corresponding to Higgs bundles.
	\item In this case, we can check that the desired result holds -- this part really requires the existence of poles / the use of a divisor $D$ with the stated properties.
\end{enumerate}

\section{10/11 (Noah Olander) -- Root Stacks and Tame Ramification}

The goal of this talk is to discuss a partial generalization of a theorem of Grothendieck on fundamental groups.

\subsection{A theorem of Grothendieck}

Let $X$ be a scheme, $\Et(X)$ the category of schemes which are \'etale over $X$, and $\FEt(X)$ the category of schemes which are finite \'etale over $X$.
If $(R, \mfr)$ is a local ring and $X$ is an $R$-scheme, we will write $X_n = X \times_{\Spec R} \Spec(R / \mfr^{n+1})$.

\begin{thm}
	Let $R$ be a complete noetherian local ring.
	Suppose $X$ is proper over $R$.
	Then the pullback map $\FEt(X) \to \FEt(X_0)$ is an equivalence.
\end{thm}

As a consequence, if $X$ and $X_0$ are connected and $\ol{x}: \Spec \ol{k}  \to X_0$ is a geometric point, then $\pi_1(X_0, \ol{x}) = \pi_1(X_0, \ol{x})$.
We'll try to generalize this statement, but first we'll talk about the proof.

\begin{proof}
	There are two main ingredients:
	\begin{enumerate}
		\item \emph{Grothendieck's existence theorem}: in the situation above, the natural map $\Coh(X) \to \lim_n \Coh(X_n)$ is an equivalence.
			Here $\lim_n \Coh(X_n)$ is the category of systems $(\Fc_n \in \Coh(X_n), \alpha_n: \Fc_{n+1} \xrightarrow{\sim} \Fc_n)$.
			This really requires completeness of $R$ and properness of $X$.
		\item If $i: Z \hookrightarrow X$ is a closed immersion with $\Ic_Z^2 = 0$, then $\Et(X) \to \Et(Z)$ is an equivalence.
			One can see that this functor is fully faithful by the formal criterion for \'etaleness.
			Once we know the functor is fully faithful, the question of essential surjectivity reduces to a local question, and we can do this explicitly by lifting presentations.
	\end{enumerate}
	Given these, we obtain $\FEt(X) \xrightarrow{\sim} \lim_n \FEt(X_n) \xrightarrow{\sim} \FEt(X_0)$.
\end{proof}

The theorem and its proof generalize easily to algebraic stacks which are proper over $R$.
We'd like find a partial generalization to the non-proper case.

\subsection{Root stacks}

Our generalization will involve root stacks.

Let's consider the local situation first: let $X$ be a scheme, $f \in \Oc(X)$.
Suppose $r \in \ZZ_{\geq 1}$ is invertible in $\Oc(X)$.
Write $X_r = \ul{\Spec} (\Oc_X[t] / (t^r - f))$.
If $T$ is a scheme over $X$, then
\[
	X_r(t) = \{ g \in \Oc(T) \,|\, g^r = f \}
\]
has an action of $\mu_r(T)$ via $a \cdot g = ag \in \Oc(T)$.
The \emph{root stack} is $\Xfr_r = [X_r / \mu_r]$.

Globally, let $\Lc \in \Pic(X)$ and $s \in \Gamma(X, \Lc)$.
Continue to assume $r$ is invertible in $\Oc(X)$.
We define a \emph{root stack} $\Xfr_r$ over $X$ with $T$-points (for $f: T \to X$) given by
\[
	\Xfr_r(T) = \big\{ \big(\Mc \in \Pic(T), \alpha: f^* \Lc \xrightarrow{\sim} \Mc^{\otimes r}, t \in \Gamma(T, \Mc)\big) \,\big|\, \alpha(f^* s) = t^r\big\}
\]
If $\Lc$ is trivial, then this agrees with the local construction above.
In particular, it follows that $\Xfr_r$ is algebraic.

Note that $\Xfr_r$ has a tautological line bundles $\Mc$ and section $t$ such that (writing $p: \Xfr_r \to X$ for the structure morphism) we have $\Mc^{\otimes r} \cong p^* \Lc$ and $t^r = s$.

\begin{ex}
	Let $X = \AA^1_\CC = \Spec \CC[x]$, and let $f = x$.
	Then $X_2 = \Spec k[x, t] / (t^2 - x)$, and $\Xfr_2 = [X_2 / \mu_2]$ looks like $\AA^1_\CC$ with a copy of $B\mu_2$ at the origin.
\end{ex}

\begin{lem}
	Let $D = V(s)$.
	\begin{enumerate}
		\item If $r \in \Oc(X)^\times$, then $\Xfr_r$ is a Deligne-Mumford stack.
		\item Let $U = X \setminus D$.
			Then $\Xfr_r \to X$ is an isomorphism over $U$.
		\item Let $\Dfr_r = V(t)$.
			Then $\Dfr_r$ is a $\mu_r$-gerbe over $D$.
		\item If $s: \Oc_X \hookrightarrow \Lc$, then $t: \Oc_{\Xfr_r} \hookrightarrow \Mc$.
		\item If $s: \Oc_X \hookrightarrow \Lc$ and $X$ and $D$ are regular noetherian schemes, then $\Xfr_r$ and $\Dfr_r$ are regular.
	\end{enumerate}
\end{lem}

\begin{proof}
	Everything is local, so we reduce to the case $\Lc = \Oc_X$.
	\begin{enumerate}
		\item The map $X_r \to \Xfr_r$ is an \'etale cover by a scheme.
		\item The map $U \times_X X_r \to U$ is a $\mu_r$-torsor.
			Now observe that, if $R$ is a ring and $a \in R^\times$, then $\mu_r(R)$ acts simply transitively on the set of roots $\{t \,|\, t^r = a\}$.
		\item This follows from a local computation together with the fact that $D$ has trivial $\mu_r$ action.
		\item Left as an exercise.
		\item Left as an exercise. \qedhere
	\end{enumerate}
\end{proof}

\subsection{Tame ramification}

Let $A$ be a DVR and $K = \Frac(A)$.
Let $L / K$ be a finite separable extension, and let $B$ be the integral closure of $A$ in $L$.
Let $\mfr_1, \dots, \mfr_s$ denote the maximal ideals of $B$, and let $e_i$ be the ramification index of $A \to B_{\mfr_i}$ (i.e.\ the integers such that $\pi_A \mapsto u \pi_B^{e_i}$ where $u \in B^\times$).

\begin{dfn}
	The extension $L / K$ is:
	\begin{enumerate}
		\item \emph{unramified} with respect to $A$ if $e_i = 1$ for all $i$.
		\item \emph{tamely ramified} with respect to $A$ if $e_i \in A^\times$ for all $i$.
			(Equivalently, $\charop A / \mfr_A$ does not divide $e_i$ for any $i$.)
	\end{enumerate}
\end{dfn}

\begin{lem}[Abhyankar]
	If $L / K$ is tamely ramified with respect to $A$, $\pi \in A$ is a uniformizer, and $r \in \ZZ$ is such that $e_i | r$ for all $i$ and $r \in A^\times$, then $L[\pi^{1/r}] / K[\pi^{1/r}]$ is unramified with respect to $A[T] / (T^r - \pi)$.
\end{lem}

Let $X$ be an integral regular noetherian scheme.
Let $D$ be an effective divisor on $X$ with generic points $\eta_1, \dots, \eta_m$.
Write $U = X \setminus D$.
Let $f: Y \to U$ be finite \'etale with $Y$ integral, and write $L / K$ for the corresponding extension of function fields.

\begin{dfn}
	In the above setup, we say $f$ is \emph{unramified} (resp.\ \emph{tamely ramified}) along $D$ if $K / L$ is so with respect to $\Oc_{X,\eta_i}$ for all $i$.
\end{dfn}

\begin{prop}[Generalized Abhyankar's Lemma]
	Let $X$ be a regular scheme over a field $k$, and let $D \subset X$ be a regular divisor.
	Let $f: Y \to X$ be tamely ramified along $D$.
	Then there exists $n \in \ZZ_{\geq 1}$ invertible in $k$ such that $f$ extends to a finite \'etale morphism $\Yfr \to \Xfr_r$ (where $Y \subset \Yfr$ is a dense open).
\end{prop}

We're out of time, but we can at least state the theorem we've been building up to.

\begin{thm}
	Let $R$ be a regular noetherian complete local ring.
	Let $X / R$ be smooth and proper, and let $D \subset X$ be a smooth effective Cartier divisor.
	Let $U = X \setminus D$.
	Then any finite \'etale morphism $Y_0 \to U_0$ which is tamely ramified along $D_0$ extends to a finite \'etale morphism $Y \to U$ which is tamely ramified along $D$.
\end{thm}

\end{document}
